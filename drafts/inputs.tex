\RequirePackage{snapshot} % just so I can automatically copy required files


%\graphicspath{{./figures/}{G:/veneto/davidbarber/courses/mlgm/cup/figures/}{G:/veneto/davidbarber/courses/mlgm/cup_first_edition/figures/}}



\usepackage{tikz,amssymb,array,delarray,verbatim,alltt,epsfig,float,amsthm}
\usepackage{graphicx,algorithmicx,algorithm,algpseudocode,xspace,xcolor}
\usepackage{url}
\usepackage{epsf,epsfig,subfigure,dsfont}
\usepackage{amssymb,amsmath}
\usepackage{pgf,amsbsy,amsfonts,amsmath,subfigure,multicol,natbib}
\usepackage{graphicx,xspace,color,cancel}
\usepackage{tikz}
\usetikzlibrary{arrows,shapes,backgrounds,through,shadows,snakes}
\usetikzlibrary{decorations.pathmorphing}
\usetikzlibrary{calc}
\usetikzlibrary{matrix}

\graphicspath{{./images/}{./figures/}{../matlab/}{../matlab/figures/}{../cup_first_edition/figures/}{./../codeOO/misc/}{../cup_first_edition_corrections/figures/}{./../codeOO/DemosExercises/}}


\def\firstcircle{(-0.25,0) circle (0.5)}
\def\secondcircle{(0.25,0) circle (0.5)}
\def\thirdcircle{(0.75,0) circle (0.5)}
%\begin{tikzpicture}[fill opacity=0.5,scale=0.5]
%\fill[red] \firstcircle;
%\fill[blue] \secondcircle;
%\fill[green]\thirdcircle;
%\end{tikzpicture}

% TIKZ stuff:

\tikzstyle{bigdot}=[circle,draw=blue!80,fill=blue!25,anchor=center,align=center,minimum size=6mm]
\tikzstyle{bigemptydot}=[circle,draw=blue!80,anchor=center,align=center,minimum size=6mm]

\tikzstyle{biggreendot}=[circle,draw=green!80,fill=green!25,anchor=center,align=center,minimum size=6mm]
\tikzstyle{bigreddot}=[circle,draw=red!80,fill=red!25,anchor=center,align=center,minimum size=6mm]

\tikzstyle{dot}=[circle,draw=orange!80,fill=orange!25,anchor=center,align=center,minimum size=4mm]
\tikzstyle{emptydot}=[circle,draw=orange!80,anchor=center,align=center,minimum size=4mm]

\tikzstyle{reddot}=[circle,draw=red!80,fill=red!25,anchor=center,align=center,minimum size=4mm]

\tikzstyle{greendot}=[circle,draw=green!80,fill=green!25,anchor=center,align=center,minimum size=4mm]

\tikzstyle{bluedot}=[circle,draw=blue!80,fill=blue!25,anchor=center,align=center,minimum size=4mm]

\tikzstyle{neur}=[rectangle,draw=green!50,fill=green!50,minimum size=6mm,line width=2pt,>=stealth]  % continuous

% my basic style definitions for Graphical Models
\newcommand{\betadist}[3]{B\br{#1|#2,#3}}

\tikzstyle{fact}=[fill,minimum size=1.5mm,line width=2pt,>=stealth]

\tikzstyle{cont2}=[circle,draw=black!50,top color=white, % a shading that is white at the top...
bottom color=lilla!50!black!20,thick,minimum size=6mm,line width=1pt,>=stealth]  % continuous  node

\tikzstyle{contredb}=[circle,draw=red,top color=red, % a shading that is white at the top...
bottom color=red!50!black!20,thick,minimum size=8.1mm,line width=1pt,>=stealth]  % continuous  node
\tikzstyle{contyellowb}=[circle,draw=yellow,top color=yellow, % a shading that is white at the top...
bottom color=yellow!50!black!20,thick,minimum size=8.1mm,line width=1pt,>=stealth]  % continuous  node
\tikzstyle{contblueb}=[circle,draw=blue,top color=blue, % a shading that is white at the top...
bottom color=blue!50!black!20, thick,minimum size=8.1mm,line width=1pt,>=stealth]  % continuous  node

\tikzstyle{contred}=[circle,draw=red,top color=red, % a shading that is white at the top...
bottom color=white,thick,minimum size=6mm,line width=1pt,>=stealth]  % continuous  node
%\tikzstyle{contred}=[circle,draw=red,top color=white, % a shading that is white at the top...
%bottom color=white,thick,minimum size=6mm,line width=1pt,>=stealth]  % continuous  node
\tikzstyle{contyellow}=[circle,draw=yellow,top color=yellow, % a shading that is white at the top...
bottom color=yellow!50!black!20,thick,minimum size=6mm,line width=1pt,>=stealth]  % continuous  node
\tikzstyle{contblue}=[circle,draw=blue,top color=blue!80!black!50, % a shading that is white at the top...
bottom color=white, thick,minimum size=6mm,line width=1pt,>=stealth]  % continuous  node
\tikzstyle{contgreen}=[circle,draw=green,top color=green!80!black!50, % a shading that is white at the top...
bottom color=white, thick,minimum size=6mm,line width=1pt,>=stealth]  % continuous  node
\tikzstyle{contwhite}=[circle,draw=white,color=white, thick,minimum size=6mm,line width=1pt,>=stealth]  % continuous  node
\tikzstyle{contwhiteb}=[circle,draw=white,color=white, thick,minimum size=7.5mm,line width=1pt,>=stealth]  % continuous  node

\tikzstyle{cont}=[circle, draw,% a shading that is white at the top...
thick,minimum size=6mm,line width=1pt,>=stealth]  % continuous  node

\tikzstyle{contb}=[circle,draw=black!50,top color=white, % a shading that is white at the top...
bottom color=darkgreen!50!black!20,thick,inner sep=0pt,minimum size=8.1mm,line width=1pt,>=stealth]  % continuous  node

\tikzstyle{ocont}=[ellipse,draw=blue!50,thick,minimum size=6mm,>=stealth]  % continuous  node
\tikzstyle{blackcont}=[circle,draw=black!50,thick,minimum size=6mm,line width=2pt,>=stealth]  % continuous  node
%\tikzstyle{par}=[circle,fill,left]  % parameter  node
\tikzstyle{oval}=[ellipse,draw=blue!50,thick,minimum size=6mm,line width=1pt,>=stealth]  % continuous node
\tikzstyle{ovalb}=[ellipse,draw=blue!50,thick,minimum size=7.5mm,line width=1pt,>=stealth]  % continuous node
\tikzstyle{disc}=[rectangle,draw=blue!50,thick,line width=1pt,minimum size=6mm]  % discrete node
\tikzstyle{obs}=[fill=blue!20,thick]  % observed node
\tikzstyle{opt}=[star,draw=red!50,thick,minimum size=6mm]  % decision node

\tikzstyle{fillred}=[fill=red!20,thick]  % observed node
\tikzstyle{fillgreen}=[fill=green!20,thick]  % observed node
\tikzstyle{purered}=[fill=red]  % observed node
\tikzstyle{state}=[rectangle,fill=red!20]  % state
\tikzstyle{sobs}=[fill=green!15,thick]  % sequentally observed node
\tikzstyle{fact}=[fill,minimum size=1.5mm,line width=2pt,>=stealth]
\tikzstyle{varfact}=[draw,minimum size=1.5mm,line width=2pt,>=stealth]
\tikzstyle{sep}=[rectangle,draw=magenta!50,thick,minimum size=6mm]  % discrete node
\tikzstyle{det}=[fill=red!15,rectangle,draw=red!50,thick,minimum size=6mm]  % deterministic node

\tikzstyle{dethid}=[diamond,draw=red!50,thick,minimum size=6mm]  % deterministic  hidden node

\tikzstyle{lineball}=[fill,-*,draw=red!50,line width=1.5pt]
\tikzstyle{redball}=[mark=*,mark options={fill=red!50,draw=red},mark size=0.5pt]
\tikzstyle{greenball}=[mark=*,mark options={fill=green!50,draw=green},mark size=0.5pt]
\tikzstyle{hid}=[circle,draw,thick]  %  non observed node

\tikzstyle{dec}=[rectangle,draw=red!50,thick,minimum size=6mm]  % decision node
\tikzstyle{utility}=[diamond,draw=red!50,thick,minimum size=6mm]  % utility node
\tikzstyle{contdec}=[circle,draw=blue!50,thick,fill=blue!10,line width=2pt]  % observed node after a decision
\tikzstyle{decutility}=[diamond,draw=red!50,thick,minimum size=6mm]  % utility node

% dependent styles
\tikzstyle{contobs}+=[cont]
\tikzstyle{contobs}+=[obs]
\tikzstyle{discobs}+=[disc]
\tikzstyle{discobs}+=[obs]

\tikzstyle{obsred}+=[obs]
\tikzstyle{obsred}+=[red]

%\tikzstyle{every picture}+=[remember picture]
\tikzstyle{background grid}=[draw, black!50,step=.1cm]
%\tikzstyle{dgraph}=[->,>=latex', line width=1.5pt]
\tikzstyle{dgraph}=[->, line width=1.5pt]
\tikzstyle{ugraph}=[line width=1.5pt]

\newcommand{\boxit}[4]
{
\path #2+(-#4,-#4) node(bottomleft){};
\path #3+(#4,#4) node(upperright){};
\draw[rounded corners,red!20] (bottomleft) rectangle (upperright);
%\path (#2-|#3) node(bottomright){};
\path (bottomleft-|upperright) node(bottomright){};
\path (bottomright) node[above left]{#1};
}
\newcommand{\boxitur}[4]
{
\path #2+(-#4,-#4) node(bottomleft){};
\path #3+(#4,#4) node(upperright){};
\draw[rounded corners,red!20] (bottomleft) rectangle (upperright);
%\path (#2-|#3) node(bottomright){};
\path (bottomleft-|upperright) node(bottomright){};
\path (upperright) node[below left]{#1};
}

% My commands:

\newcommand{\myhline}{\vspace{0.3cm} \hrule \vspace{0.3cm}}

\definecolor{magenta}{cmyk}{0.1,1,1,0.5}
\definecolor{darkgreen}{cmyk}{0.6,0.1,0.6,0.6}
\definecolor{pink}{cmyk}{0.1,1,1,0.1}
\definecolor{azzurro}{cmyk}{0.9333, 0.2471, 0.5569, 0.102}
\definecolor{lilla}{rgb}{0.5, 0, 0.5}
\definecolor{mygreen}{rgb}{0, 0.5, 0}
\definecolor{darkorange}{rgb}{1, 0.4, 0} % is probably not visible...
\definecolor{darkred}{rgb}{0.8, 0, 0}


\newcommand{\numsamp}{L}
\newcommand{\sampnum}{\numsamp}
\newcommand{\sampind}{l}
\newcommand{\sampindp}{{l'}}
\newcommand{\sampindpp}{{l''}}
\newcommand{\euclidean}{Euclidean\xspace}
\newcommand{\psamp}{\tilde{p}}
\newcommand{\cpt}{CPT\xspace}
\newcommand{\cpts}{CPTs\xspace}
\newcommand{\multivariate}{multi-variate\xspace}

\newcommand{\demo}[1]{{\color{red}{{\tt{#1}}}}}

\newcommand{\red}[1]{{\color{red}{#1}}}
\newcommand{\blue}[1]{{\color{blue}{#1}}}
\newcommand{\green}[1]{{\color{green}{#1}}}
\newcommand{\pink}[1]{{\color{magenta}{#1}}}


\newcommand{\bayes}{{Bayes'} }
\newcommand{\eframe}{\end{frame}}
\newcommand{\bframe}[1]{\begin{frame}{#1}}
\newcommand{\head}[1]{\vspace{2mm}\hrule\vspace{2mm}\structure{#1}\\}
\newcommand{\tophead}[1]{\structure{#1}\\}
\newcommand{\myhrule}{\vspace{2mm}\hrule\vspace{2mm}\newl}
\newcommand{\means}{$\Rightarrow$}
\newcommand{\mymagenta}{\color{magenta}}
\newcommand{\myblue}{\color{blue}}
\newcommand{\myazzurro}{\color{azzurro}}
\newcommand{\mylilla}{\color{lilla}}
\newcommand{\mygreenbf}{\bf\color{darkgreen}}
\newcommand{\mygreen}{\color{darkgreen}}
\newcommand{\myred}{\color{darkred}}
\newcommand{\myyellow}{\color{yellow}}
\newcommand{\mywhite}{\color{white}}
\newcommand{\myblack}{\color{black}}

\def\ci{\perp\!\!\!\perp}

\newcommand{\tw}{\textwidth}
\newcommand{\newl}{\hspace{1mm}\\}
\newcommand{\bmp}[1]{\begin{minipage}{#1}}
\newcommand{\bmpp}[2]{\begin{minipage}[#1]{#2}}
\newcommand{\emp}{\end{minipage}}
\newcommand{\cb}[1]{\left\{ {#1} \right\}}
\newcommand{\br}[1]{\left( {#1} \right)}
\newcommand{\sq}[1]{\left[ {#1} \right]}

\renewcommand{\v}[1]{\mathbf{#1}}

\newcommand{\m}[1]{\mathbf{#1}}
\newcommand{\fphi}[2]{f_{#1}\br{#2}}
\newcommand{\fxmess}[2]{\mu_{{#1\rightarrow #2}}\left( #2 \right)}
\newcommand{\xfmess}[2]{\mu_{{#1\rightarrow #2}}\left( #1 \right)}
\newcommand{\panel}[1]{ \textbf{(#1)}:}

\newcommand{\av}[1]{\left\langle{#1}\right\rangle}
\newcommand{\argmax}[1]{\ensuremath{\underset{#1}{\mathrm{argmax}\ }}}
\newcommand{\argmin}[1]{\ensuremath{\underset{#1}{\mathrm{argmin}\ }}}
\newcommand{\sfs}{\fontsize{18pt}{18pt}}

\newcommand{\Id}{\m{I}}
\newcommand{\ndist}[3]{{\cal{N}}\br{#1\thinspace\vline\thinspace #2,#3}}
\newcommand{\vtheta}{\alpha}
\def\ci{\perp\!\!\!\perp}
\def\dep{\top\!\!\!\top}
\newcommand{\bfx}{{\v{x}}}
\newcommand{\sta}[1]{\textsf{#1}}
\newcommand{\st}[1]{\sta{#1}}
\newcommand{\va}[1]{\lowercase{#1}}
\newcommand{\dom}[1]{\textrm{dom}\!\br{#1}}
\newcommand{\sett}[1]{\myset{#1}}
\newcommand{\ocm}{\hspace{1cm}}
\newcommand{\hcm}{\hspace{0.25cm}}
\newcommand{\ie}{{\em i.e. \xspace}}
\newcommand{\eg}{{\em e.g. \xspace}}
\newcommand{\etc}{{\em etc. \xspace}}
\newcommand{\wrt}{{\em w.r.t. \xspace}}
\newcommand{\aka}{{\em aka. \xspace}}
\newcommand{\wlogen}{without loss of generality\xspace}
\newcommand{\lhs}{left hand side\xspace}
\newcommand{\rhs}{right hand side\xspace}
\newcommand{\event}[1]{{#1}}
\newcommand{\cindep}[3]{#1\!\ci\!#2\!\!\mid\!#3}
\newcommand{\cdep}[3]{#1\dep #2\!\!\mid\!#3}
\newcommand{\indep}[2]{#1\!\ci\!#2}
\newcommand{\depp}[2]{#1\!\dep\!#2}
\newcommand{\myset}[1]{\mathcal{\uppercase{#1}}}
\newcommand{\beq}{\begin{equation}}
\newcommand{\eeq}{\end{equation}}
%\newcommand{\beq}{\[}
%\newcommand{\eeq}{\]}
\newcommand{\nn}{{\nonumber}}
\newcommand{\datacount}[1]{\sharp\br{#1}}
\newcommand{\datacountc}{\sharp}
\newcommand{\datacountC}{\sharp}
\newcommand{\maxlik}{Maximum Likelihood\xspace}
\newcommand{\pa}[1]{{\rm{pa}}\left({#1}\right)}
\newcommand{\ch}[1]{{\rm{ch}}\left({#1}\right)}

\newcommand{\evalat}[2]{\left.{#1}\right\vert_{#2}}


\newcommand{\kl}[2]{\textrm{KL}\!\br{#1|#2}}
\newcommand{\numv}{K}
\newcommand{\KL}{Kullback-Leibler\xspace}

\newcommand{\tenp}[1]{\times 10^{#1}}
\newcommand{\aposteriori}{\emph{a posteriori}\xspace}
\newcommand{\apriori}{\emph{a priori}\xspace}
\newcommand{\mtongue}{{MT}}
\newcommand{\english}{\st{Eng}}
\newcommand{\welsh}{\st{Wel}}
\newcommand{\scottish}{\st{Scot}}
\newcommand{\country}{Cnt}
\newcommand{\england}{\st{E}}
\newcommand{\wales}{\st{W}}
\newcommand{\scotland}{\st{S}}
\newcommand{\vc}[4]{\begin{array}({c})#1\\#2\\#3\\#4\end{array}}
\newcommand{\vcc}[2]{\begin{array}({c})#1\\#2\end{array}}
\newcommand{\vccc}[3]{\begin{array}({c})#1\\#2\\#3\end{array}}
\newcommand{\vcccc}[4]{\begin{array}({c})#1\\#2\\#3\\#4\end{array}}
\newcommand{\kernel}{K}
\newcommand{\dataind}{n}
\newcommand{\dataindp}{{n'}}
\newcommand{\dataindm}{{m}}
\newcommand{\Dataind}{N}
\newcommand{\datadim}{D}
\newcommand{\basisdim}{B}
\newcommand{\data}{{\cal{D}}}
\newcommand{\myemph}[1]{{\color{red}\emph{#1}}}
\newcommand{\mem}[1]{{\color{red}\emph{#1}}}
\newcommand{\iid}{i.i.d.\xspace}

\newcommand{\const}{\textsf{const.}}
\newcommand{\tru}{\textsf{tr}}
\newcommand{\fal}{\textsf{fa}}
\newcommand{\abs}[1]{|#1|}
\newcommand{\minf}[2]{\textrm{MI}\!\br{#1;#2}}
\newcommand{\bigo}[1]{O\br{#1}}


\newcommand{\figref}[1]{{figure(\ref{#1})}}


\newcommand{\cpot}[1]{{\phi\left( #1\right)}}
\newcommand{\cpots}[1]{{\phi^*\left( #1\right)}}
\newcommand{\cpotss}[1]{{\phi^{**}\left( #1\right)}}
\newcommand{\cpotp}[1]{{\hat{\phi}\left( #1\right)}}
\newcommand{\minus}{\backslash}
\newcommand{\union}{\cup}
\newcommand{\dirichlet}[2]{{\rm Dirichlet}\br{#1|#2}}
\newcommand{\jpai}{pa(i)}
\newcommand{\insect}{\cap}
\newcommand{\bueq}{\[}
\newcommand{\eueq}{\]}
\newcommand{\bnet}{belief network\xspace}
\newcommand{\minv}[1]{{\boldsymbol{{\rm #1}}^{-1}}}
\newcommand{\bnets}{belief networks\xspace}
\newcommand{\drawsep}[4]{
\node[sep] (#1) at ($(#2)!0.5!(#3)$) {#4};
\draw(#1)--(#2);\draw(#1)--(#3);}
\tikzstyle{celim}=[circle,draw=red!25,thick,minimum size=6mm,line width=2pt,>=stealth]  %
\tikzstyle{delim}=[draw=red!25]  %
\newcommand{\vdim}[1]{\textrm{dim}\br{#1}}
\newcommand{\dimmodel}{K}
\newcommand{\btz}{\begin{tikzpicture}}
\newcommand{\etz}{\end{tikzpicture}}
\newcommand{\ind}[1]{\mathbb{I}\sq{#1}}
\newcommand{\trans}{^{\textsf{T}}}
\newcommand{\nin}{\not\in}
\newcommand{\boldDelta}{\boldsymbol{\Delta}}
\newcommand{\bolddelta}{\boldsymbol{\delta}}
\newcommand{\boldmu}{\boldsymbol{\mu}}
\newcommand{\boldeta}{\boldsymbol{\eta}}
\newcommand{\boldnu}{\boldsymbol{\nu}}
\newcommand{\boldalpha}{\boldsymbol{\alpha}}
\newcommand{\boldbeta}{\boldsymbol{\beta}}
\newcommand{\boldepsilon}{\boldsymbol{\epsilon}}
\newcommand{\boldSigma}{\boldsymbol{\Sigma}}
\newcommand{\boldsigma}{\boldsymbol{\sigma}}
\newcommand{\boldtau}{\boldsymbol{\tau}}
\newcommand{\boldlambda}{\boldsymbol{\lambda}}
\newcommand{\boldLambda}{\boldsymbol{\Lambda}}
\newcommand{\boldpsi}{\boldsymbol{\psi}}
\newcommand{\boldxi}{\boldsymbol{\xi}}
\newcommand{\boldPsi}{\boldsymbol{\Psi}}
\newcommand{\boldphi}{\boldsymbol{\phi}}
\newcommand{\boldPhi}{\boldsymbol{\Phi}}
\newcommand{\boldtheta}{\boldsymbol{\theta}}
\newcommand{\boldTheta}{\boldsymbol{\Theta}}
\newcommand{\boldgamma}{\boldsymbol{\gamma}}
\newcommand{\boldpi}{\boldsymbol{\pi}}

\newcommand{\indepdiff}{\textsf{indep}}
\newcommand{\indepsame}{\textsf{same}}
\newcommand{\depen}{\textsf{dep}}
\newcommand{\modv}[1]{|\boldsymbol{#1} |}
\newcommand{\ipv}[2]{\boldsymbol{{\rm #1}}\cdot \boldsymbol{{\rm #2}}}
\newcommand{\cpv}[2]{\boldsymbol{{\rm #1}}\times\boldsymbol{{\rm #2}}}
\renewcommand{\mod}[1]{\left| #1 \right|}  % \mod{foo} = |foo|
\newcommand{\trace}[1]{{\rm trace}\left({#1}\right)}
\newcommand{\twomat}[4]{\begin{pmatrix} #1 & #2 \\ #3 & #4\end{pmatrix}}
\newcommand{\threemat}[9]{\begin{pmatrix} #1 & #2 & #3\\ #4 & #5 & #6\\ #7 & #8 & #9\end{pmatrix}}
\newcommand{\semidefinite}{semidefinite\xspace}
\renewcommand{\det}[1]{\mathrm{det}\br{#1}}
\newcommand{\pdsd}[3]{\frac{\partial^2 #1}{\partial {#2} \partial {#3} }}
\newcommand{\pdol}[2]{\partial {#1}/ \partial {#2}}
%\newcommand{\pd}[1]{\partial #1}
\newcommand{\pdt}[2]{\frac{\partial^2 #1}{\partial {#2}^2 }}
\newcommand{\pdo}[2]{\frac{\partial {#1}}{\partial {#2}}}
\newcommand{\pdu}[1]{\frac{\partial }{\partial {#1}}}
\newcommand{\ddu}[1]{\frac{d}{d {#1}}}
\newcommand{\ddo}[2]{\frac{d {#1}}{d {#2}}}
\newcommand{\ddt}[2]{\frac{d^2 {#1}}{d {#2}^2}}
\newcommand{\bfg}{\v{g}}
\newcommand{\bfk}{\v{k}}
\newcommand{\bfp}{\v{p}}
\newcommand{\grad}{\nabla}
\newcommand{\pdiff}[2]{\frac{\partial{#1}}{\partial{#2}}}
\newcommand{\pdifftwo}[2]{\frac{\partial^2{#1}}{\partial{#2}^2}}
\newcommand{\pdiffu}[2]{\frac{\partial{#2}}{\partial{#1}}}
\newcommand{\pdif}[2]{\frac{\partial}{\partial{#2}}#1}
\newcommand{\pdiftwo}[2]{\frac{\partial^2}{\partial{#2}^2}#1}
\newcommand{\tdif}[2]{\frac{d}{d{#2}}#1}
\newcommand{\tdiff}[2]{\frac{d#1}{d{#2}}}
%\newcommand{\pdif}[2]{\frac{\partial^2}{\partial{#2}^2}#1}
\newcommand{\pdift}[3]{\frac{\partial^2}{\partial{#2}\partial{#3}}#1}
\newcommand{\tdift}[3]{\frac{d^2}{d{#2}d{#3}}#1}
\newcommand{\bfu}{\v{u}}
\newcommand{\inv}[1]{#1^{-1}}
\newcommand{\invt}[1]{#1^{-\textsf{T}}}
\newcommand{\invtrans}{^{-\textsf{T}}}
\newcommand{\minvt}[1]{{\boldsymbol{{\rm #1}}^{-\textsf{T}}}}
\newcommand{\vhat}[1]{{\hat{\v{#1}}}}
\newcommand{\bfzero}{{\bf 0}}
\newcommand{\app}[1]{\tilde{#1}}
\newcommand{\sgn}[1]{{\text{sign}}\br{#1}}

\newcommand{\half}{\frac{1}{2}}
\newcommand{\sampcov}{\m{S}}
\newcommand{\pca}{PCA\xspace}
\newcommand{\datapoint}{datapoint\xspace}
\newcommand{\timestep}{timestep\xspace}
\newcommand{\timesteps}{timesteps\xspace}
\newcommand{\timepoints}{timepoints\xspace}
\newcommand{\timeseries}{timeseries\xspace}
\newcommand{\Timeseries}{Timeseries\xspace}
\newcommand{\dataset}{dataset\xspace}

\newcommand{\tp}{\tilde{p}}
\newcommand{\tq}{\tilde{q}}
\newcommand{\ii}{\textsf{i}}
\newcommand{\dimH}{H}
\newcommand{\diag}[1]{{\rm{diag}}\left( #1 \right)}
\newcommand{\mv}{\bar{\m{v}}}
\newcommand{\tS}{\tilde{\m{S}}}
\newcommand{\datapoints}{datapoints\xspace}

\newcommand{\SigmauH}{\boldSigma^H}
\newcommand{\SigmauV}{\boldSigma^V}
\newcommand{\SigmaP}{\boldSigma_\pi}
\newcommand{\SigmaH}{\boldSigma_H}
\newcommand{\SigmaV}{\boldSigma_V}
\newcommand{\blkdiag}[1]{{\rm{blkdiag}}\left( #1 \right)} 